\documentclass[12pt]{article}
\usepackage[margin=1in]{geometry}
\usepackage{amsmath, amssymb, amsthm, graphicx, hyperref}
\usepackage{enumerate}
\usepackage{fancyhdr}
\usepackage{multirow, multicol}
\usepackage{tikz}
\pagestyle{fancy}
\fancyhead[RO]{Eliot Brown 2020}
\fancyhead[LO]{MA-UY 2314: Discrete Mathematics}
\usepackage{comment}
\newif\ifshow
\showfalse

\begin{document}

\begin{center}
\ifshow
  \textbf{\Large Homework 0 Solution}\\
\else
  \textbf{\Large Homework 2}\\
\fi
Due: Friday, Feb. 14, by 11:59pm,\\via Gradescope\\
%Late HW accepted until 9PM on September 8\footnote{For each hour your homework is late, 10 points (out of 100) will be deducted from your homework score.  Please submit your homework on time!}
\end{center}

\hrule

\vspace{0.2cm}
\noindent
\begin{enumerate}[A.]
\item Refer to the Gradescope videos on YouTube if you have questions on how to submit homework.  
\item I refer you to the syllabus for the rubric that is used on the homework.  
\end{enumerate}

\noindent 1.  Let $x$ be an integer.  We say that $x$ is a perfect square iff $x = a^{2}$ for some $a \in \mathbb{Z}$.  We say that integers $x$ and $y$ are consecutive perfect squares iff $x= a^{2}$ and $y = (a + 1)^{2}$ for some $a \in \mathbb{Z}$.
\vspace{.05in}
\begin{enumerate}
    \item [(a)] Disprove: \textit{If x and y are consecutive perfect squares, then $x - y$ is even}. \\  \\
    \textbf{Solution} $a = 2$ and $x = 4$ and $y = 9$ therefore, $x - y = 4 - 9 = -5$ odd  
\item[(b)] Prove: \textit{If x and y are consecutive perfect squares, then $x - y$ is odd}.

\noindent \textbf{Remark:} Your homework should start off as follows:
\vspace{.05in}

\noindent \textbf {Pf:}  Assume $x$ and $y$ are consecutive perfect squares.\\
\indent $x = a^2$ and $y =(a+1)^2$ \\
\indent $x - y = a^2 - (a +1)^2$ \\
\indent $x - y = a^2 - (a^2 +2a +1)$ \\ 
\indent $ x - y = a^2 - a^2 - 2a -1$ \\ 
\indent $ x - y = -2a - 1$ \\
\indent $ x - y = - (2a+1)$ - The definition of odd\\
$\square$

\end{enumerate}
\vspace{.05in}


\noindent 2.  Let $a, b, c, d, e$ be integers.  Prove: \textit{If $a | d$ and $a | e$ then $a |( db + ec)$}.
\vspace{.05in}


\noindent \textbf{Remark:} Your homework should start off as follows:
\vspace{.05in}

\noindent \textbf{Pf:}  Assume $a|d$ and $a |e$.\\
\indent Assume $a | d$ \\ 
\indent $d = k_1a$
\indent \hspace{1cm} assume $k_1, k_2 \epsilon \mathbb{Z}$ \\
\indent$e = k_2a$ \\ 
\indent$db + ec = k_1ab + k_2ac$ \\ 
\indent$db + ec = a (k_1b + k_2c)$ \hspace{1cm} where m = $k_1b + k_2c$ \\ 
\indent$db + ec = am$ \\
\indent$a|db + ec$ \\
$\square$

\vspace{.15in}

\noindent 3.  Prove \textit{ If $a$ is an integer, then $a^{2} - a$ is even}.
\vspace{.05in}

\noindent \textbf{Remark:} Your homework should start off as follows:
\vspace{.05in}

\noindent \textbf{Pf:}  Assume $a$ is an integer. \\
\indent (1) If a is odd  \\
\indent $a = 2k + 1$\\
\indent $a^2 - a = (2k+1)^2 - 2k+1$\\
\indent $a^2 - a = (4k^2 + 4k + 1) - 2k + 1$\\
\indent $a^2 - a = 4k^2 + 2k$\\
\indent $a^2 - a = 2(2k^2 + k)$\\
\indent $2|a^2 - a$\\

\indent (2) If a is even \\ 
\indent a = 2k \\
\indent $a^2 = 4k^2$ \\ 
\indent a = 2k \\
\indent $a^2 - a = 4k^2 -2k$ \hspace{1cm} $m = 2k^2 -k$\\
\indent $a^2 - a = 2m$ \\
$\square$
\vspace{.15in}


\noindent 4.  Suppose a student has proven that $x \rightarrow y $= T and $-y \rightarrow -x$ = T.  Has student proven that $x \leftrightarrow y$ = T. Explain.  \\ 
In order to prove $x \leftrightarrow y$ \\
\indent NTP: \\
\indent \indent 1) $x \rightarrow y$ \\
\indent \indent 2) $y \rightarrow x$\\

\indent student proved $x \rightarrow y$ = T \\ 
\indent student proved $-y \rightarrow -x$ = T \\ 

\indent NTP: 2) $y \rightarrow x$ = T \\ 

\indent Assume $ -y \rightarrown -x $ = T \\
\indent $y \vee -x$ \;\; \mbox Translate \\
\indent $-x \vee y$ \;\; \mbox Commutative \\
\indent $x \rightarrow y$ \mbox Translate \\
\indent Student proved $x \rightarrow y$ twice so therefore did not prove $x \leftrightarrow y$ \\
$\square$


 

\vspace{.15in}
\noindent 5.  Let $x$ and $y$ be integers.  We say that $x$ and $y$ have the same parity iff $x$ and $y$ are both even OR $x$ and $y$ are both odd.  
\vspace{.05in}

\noindent Prove: \textit{ If integers $x$ and $y$ have the same parity, then $x + y$ is even}
\vspace{.05in}

\noindent \textbf{Remark:} Your homework should start off as follows:
\vspace{.05in}

\noindent \textbf{Pf:}  Assume that $x$ and $y$ are integers with the same parity.  \\

\underline {Proving integers of an Even Parity} \\ 

\indent $x = 2k_1$ \\ 
\indent $y = 2k_2$ \\ 
\indent $x + y = 2k_1 + 2k_2$ \\
\indent $x + y = 2(k_1 +k_2$) \\
\indent $2 | x + y$ \\

\underline {Proving integers of an odd Parity} \\ 
\indent $x = 2k_1 + 1$ \\ 
\indent $y = 2k_2 + 1$ \\ 
\indent $x + y = (2k_1 + 1) + (2k_2 + 1)$ \\
\indent $x + y = 2k_1 + 2k_2 + 2$ \\
\indent $x + y = 2(k_1 + k_2 +1)$ \\ 
\indent $2|x+y$ \\
$\square$
\vspace{.15in}


\noindent 6.  Prove \textit{ If $x, y, z, w$ be four consecutive integers, then $xyzw$ is one less than a perfect square}. \\

\vspace{.05in}

\noindent \textbf{Remark:} Note that one less than a perfect sqaure means that $xyzy = a^{2} - 1$ for some integer $a$.  Your homework should start off as follows:
\vspace{.05in}

\noindent \textbf{Pf:}  Assume that $x, y, z, w$ are consecutive integers.   
\vspace{.15in}


\indent $x = k , y = k+1, z = k+2, w = k+3$ \\
\indent $xy = k^2 + k$ \\ 
\indent $xyz = (k^2 + k )(k+2)$ \\ 
\indent $xyz = k^3 + 3k^2 + 2k$ \\ 
\indent $xyzw = k^3 + 3k^2 + 2k (k+3) $ \\ 
\indent $xyzw = k^4 + 6k^3 + 11k^2 + 6k$ \\ 
\indent $xyzw = (k^4 + 6k^3 + 11k^2 + 6k +1) -1 $ \\ 
\indent $xyzw = (k^2 + 3k + 1)^2 -1 = a^2 - 1$ \\
$\square$


\noindent 7.  Let $a, b, c$ be integers.  Disprove:  \textit{ If $a | c$ and $b | c$ then $(a + b) | c$}.
\vspace{.05in}

\noindent \textbf{Remark:} You do not need to justify your counterexample. 
\vspace{.15in}

\textbf{Solution} $ a = 2 \wedge b = 3 \wedge c = 6$ as $2|6 \wedge 3|6$ but not $5|6$ which disproves the claim. \\


\noindent 8.  Consider the statement \textit{For all integers $n$, $n^{2} - n + 1$ is a prime number.  }  If this statement is False, provide a counterexample.  If this statement is True, prove it.  
\vspace{.15in}

\textbf{Solution} False: $n=5$ then $n^2-n+1= 21$ which is divisible by 3. \\



\noindent 9.  Prove: \textit{If $x$ is an odd integers, then the square of $x$ is of the form $8m + 1$ for some $m \in \mathbb{Z}$.}
\vspace{.05in}

\noindent \textbf{Remark:}  The statement: \textit{ the square $x$ is of the form $8m + 1$ for some $m \in \mathbb{Z}$.} is saying 
\[
x^{2} = 8m + 1, \;\; \mbox{ for some } m \in \mathbb{Z}
\]


\noindent \textbf{Remark:} Your homework should start off as follows:
\vspace{.05in}

\noindent \textbf{Pf:}  Assume that $x$ is an odd integer.  \\ 
\indent $x = 2k+1$ and $k = 2a+1$ \\ 
\indent $x^2 = (2k+1)^2$ \\
\indent $x^2 = 4k^2 + 4k + 1 = 8m + 1$ \\ 
\indent $x^2 = 4k^2 + 4k = 8m$ \\ 
\indent $x^2 = 4(k^2 + k) = 8m$ \\
\indent $x^2 = k^2 + k = 2m$ \\
\indent $x^2 = ((2a+1)^2 +(2a+1))= 2m$\\
\indent $x^2 = 4a^2 + 6a + 1 + 2a + 1= 2m$\\
\indent $x^2 = 4a^2 +8a + 2= 2m$ \\
\indent $x^2 = 2(a^2 + 4a +1)= 2m$\\
\indent $ m = a^2 + 4a + 1$\\
$\square$
\vspace{.15in}









\end{document}

