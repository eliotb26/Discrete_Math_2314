\documentclass[12pt]{article}
\usepackage[margin=1in]{geometry}
\usepackage{amsmath, amssymb, amsthm, graphicx, hyperref}
\usepackage{enumerate}
\usepackage{fancyhdr}
\usepackage{multirow, multicol}
\usepackage{tikz}
\pagestyle{fancy}
\fancyhead[RO]{Spring 2020}
\fancyhead[LO]{MA-UY 2314: Discrete Mathematics}
\usepackage{comment}
\newif\ifshow
\showfalse

\begin{document}

\begin{center}
\ifshow
  \textbf{\Large Homework 0 Solution}\\
\else
  \textbf{\Large Homework 2}\\
\fi
Due: Friday, Feb. 14, by 11:59pm,\\via Gradescope\\
%Late HW accepted until 9PM on September 8\footnote{For each hour your homework is late, 10 points (out of 100) will be deducted from your homework score.  Please submit your homework on time!}
\end{center}

\hrule

\vspace{0.2cm}
\noindent
\begin{enumerate}[A.]
\item Refer to the Gradescope videos on YouTube if you have questions on how to submit homework.  
\item I refer you to the syllabus for the rubric that is used on the homework.  
\end{enumerate}

\noindent 1.  Let $x$ be an integer.  We say that $x$ is a perfect square iff $x = a^{2}$ for some $a \in \mathbb{Z}$.  We say that integers $x$ and $y$ are consecutive perfect squares iff $x= a^{2}$ and $y = (a + 1)^{2}$ for some $a \in \mathbb{Z}$.
\vspace{.05in}
\begin{enumerate}
    \item [(a)] Disprove: \textit{If x and y are consecutive perfect squares, then $x - y$ is even}.
\item[(b)] Prove: \textit{If x and y are consecutive perfect squares, then $x - y$ is odd}.

\noindent \textbf{Remark:} Your homework should start off as follows:
\vspace{.05in}

\noindent \textbf{Pf:}  Assume $x$ and $y$ are consecutive perfect squares.  
\end{enumerate}
\vspace{.05in}


\noindent 2.  Let $a, b, c, d, e$ be integers.  Prove: \textit{If $a | d$ and $a | e$ then $a |( db + ec)$}.
\vspace{.05in}


\noindent \textbf{Remark:} Your homework should start off as follows:
\vspace{.05in}

\noindent \textbf{Pf:}  Assume $a|d$ and $a |e$.
\vspace{.15in}

\noindent 3.  Prove \textit{ If $a$ is an integer, then $a^{2} - a$ is even}.
\vspace{.05in}

\noindent \textbf{Remark:} Your homework should start off as follows:
\vspace{.05in}

\noindent \textbf{Pf:}  Assume $a$ is an integer.  
\vspace{.15in}


\noindent 4.  Suppose a student has proven that $x \rightarrow y $= T and $-y \rightarrow -x$ = T.  Has student proven that $x \leftrightarrow y$ = T. Explain.  
\vspace{.15in}

\noindent 5.  Let $x$ and $y$ be integers.  We say that $x$ and $y$ have the same parity iff $x$ and $y$ are both even OR $x$ and $y$ are both odd.  
\vspace{.05in}

\noindent Prove: \textit{ If integers $x$ and $y$ have the same parity, then $x + y$ is even}
\vspace{.05in}

\noindent \textbf{Remark:} Your homework should start off as follows:
\vspace{.05in}

\noindent \textbf{Pf:}  Assume that $x$ and $y$ are integers with the same parity.  
\vspace{.15in}


\noindent 6.  Prove \textit{ If $x, y, z, w$ be four consecutive integers, then $xyzw$ is one less than a perfect square}. 
\vspace{.05in}

\noindent \textbf{Remark:} Note that one less than a perfect sqaure means that $xyzy = a^{2} - 1$ for some integer $a$.  Your homework should start off as follows:
\vspace{.05in}

\noindent \textbf{Pf:}  Assume that $x, y, z, w$ are consecutive integers.   
\vspace{.15in}


\noindent 7.  Let $a, b, c$ be integers.  Disprove:  \textit{ If $a | c$ and $b | c$ then $(a + b) | c$}.
\vspace{.05in}

\noindent \textbf{Remark:} You do not need to justify your counterexample. 
\vspace{.15in}


\noindent 8.  Consider the statement \textit{For all integers $n$, $n^{2} - n + 1$ is a prime number.  }  If this statement is False, provide a counterexample.  If this statement is True, prove it.  
\vspace{.15in}

\noindent 9.  Prove: \textit{If $x$ is an odd integers, then the square of $x$ is of the form $8m + 1$ for some $m \in \mathbb{Z}$.}
\vspace{.05in}

\noindent \textbf{Remark:}  The statement: \textit{ the square $x$ is of the form $8m + 1$ for some $m \in \mathbb{Z}$.} is saying 
\[
x^{2} = 8m + 1, \;\; \mbox{ for some } m \in \mathbb{Z}
\]


\noindent \textbf{Remark:} Your homework should start off as follows:
\vspace{.05in}

\noindent \textbf{Pf:}  Assume that $x$ is an odd integer.   
\vspace{.15in}









\end{document}