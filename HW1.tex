\documentclass[12pt]{article}
\usepackage[margin=1in]{geometry}
\usepackage{amsmath, amssymb, amsthm, graphicx, hyperref}
\usepackage{enumerate}
\usepackage{fancyhdr}
\usepackage{multirow, multicol}
\usepackage{tikz}
\pagestyle{fancy}
\fancyhead[RO]{Eliot Brown}
\fancyhead[LO]{MA-UY 2314: Discrete Mathematics}
\usepackage{comment}
\newif\ifshow
\showfalse

\begin{document}

\begin{center}
\ifshow
  \textbf{\Large Homework 0 Solution}\\
\else
  \textbf{\Large Homework 1}\\
\fi
Due: Friday, Feb. 7, by 11:59pm,\\via Gradescope\\
%Late HW accepted until 9PM on September 8\footnote{For each hour your homework is late, 10 points (out of 100) will be deducted from your homework score.  Please submit your homework on time!}
\end{center}

\hrule

\vspace{0.2cm}
\noindent
\begin{enumerate}[A.]
\item Refer to the Gradescope videos on YouTube if you have questions on how to submit homework.  
\item I refer you to the syllabus for the rubric that is used on the homework.  
\item To access Gradescope vis NYU Classes, click the Lessons tab followed by the Gradescope link. 
\end{enumerate}


\noindent \textbf{Section 3 Problems:} 
\vspace{.15in}

\noindent 1.(24 points) 3.1. See the latex help file for examples on how to explain your answer. 
\vspace{.15in}

% A 
\noindent (a) $3|100$ = \textbf{F}.  There is no $k \in \mathbb{Z}$ such that $100 = 3k$.
%B
\newline (b) $3|99$= \textbf{T}.  Notice that $99 = 3k$ where $k = 33$. 
%C 
\newline (c) $-3|3$ = \textbf{T}. This is because $3 = -3k$ where $k= -1$. 
%D
\newline (d) $-5|-5$ = \textbf{T}. This is because $-5 = -5k$ where $k = 1$. 
%E 
\newline (e) $-2|-7$ = \textbf{F}. There is no $k \in \mathbb{Z}$ where $-7 = -2k$. 
%F
\newline (f) $0|4$ = \textbf{F}. There is no $k \in \mathbb{Z}$ where $4 = 0k$. 
%G
\newline (g) $4|0$ = \textbf{T}. This is because $0 = 4k$ where $k=0$. 
%H 
\newline (h) $0|0$ = \textbf{T}. This is because $0 = 0k$ where $k= \in \mathbb{Z} $
\vspace{.15in}

%QUESTION 2
\noindent 2.(3 points) 3.5
\vspace{.15in}
\newline
Explanation: Given that integers are all whole numbers, positive and negative, and 0. Rational numbers can be represented as fractions in between the integers, such as $1/2$, of the form of $a/b$ where $a,b \in \mathbb{Z}$. As rational numbers can also be whole numbers, positive and negative, then all integers are rational numbers. But as $1/2$ is a fraction in between two integers, it proves that not all rational numbers are integers. 
\vspace{0.25in}

\noindent 3.(6 points) 3.12  (k), (l)
\vspace{.25in}

(k) divisors = 96

(L) divisors = Infinitely many, {0,1,2,...n}

\vspace{.25in}



%Section 7 
\noindent \textbf{Section 7 Problems:} 
\vspace{.15in}

%4
\noindent 4.(3 points) 7.10(c)    
\vspace{.15in}

(c) 
\[ %centers the table.
\begin{tabular}{|c|c|c|c|c|c|c|c|} 
\hline %\hline gives us a horizontal line.  
$x$ & $y$ & $x\vee y$ & $-y$ & $x\wedge -y$ & $-x$ & $-x\wedge y$ & $(x \wedge -y) \vee (-x \wedge y)$ \\ 
\hline
T&T&T&F&F&F&F&F \\
\hline 
T&F&T&T&T&F&F&T\\
\hline
F&T&T&F&F&T&T&T\\
\hline
F&F&F&T&F&T&F&F \\
\hline
\end{tabular} %end of the table.  
\] %need to close of the center of the table.  
This shows that the two are not the same through the evidence of a truth table 
\vspace{0.25in}



\noindent 5.(3 points) 7.5.  Use a truth table when solving this problem.      
\vspace{.15in}

\[ %centers the table.
\begin{tabular}{|c|c|c|c|c|c|} 
\hline %\hline gives us a horizontal line.  
$x$ & $y$ & $-x$ & $-y$ & $x\leftrightarrow y$ & $-x \leftrightarrow -y$ \\ 
\hline
T&T&F&F&T&T \\
\hline 
T&F&F&T&F&F \\
\hline
F&T&T&F&F&F \\
\hline
F&F&T&T&T&T \\
\hline
\end{tabular} %end of the table.  
\] %need to close of the center of the table.  




\noindent 6.(3 points) 7.8.  Use a truth table when solving this problem. 
\vspace{.15in}

\[ %centers the table.
\begin{tabular}{|c|c|c|c|c|c|c|c|} 
\hline %\hline gives us a horizontal line.  
$x$ & $y$ & $z$ & $x \vee y$ & $(x \vee y) \rightarrow z$ & $x\rightarrow z$ & $y \rightarrow z$ & $(x \rightarrow z) \wedge (y \rightarrow z)$ \\ 
\hline
T&T&T&T&T&T&T&T \\
\hline
T&T&F&T&F&F&F&F \\
\hline 
T&F&T&T&T&T&T&T \\
\hline
F&T&T&T&T&T&T&T \\
\hline
T&F&F&T&F&F&T&F \\
\hline
F&F&T&F&T&T&T&T \\
\hline 
F&T&F&T&F&T&F&F \\
\hline
F&F&F&F&T&T&T&T \\
\hline

\end{tabular} %end of the table.  
\] 
\vspace{0.5in}



\noindent 7.(9 points) 7.11 (a), (b), (h).  Use Theorem 7.2 when proving these problems.  Note that part (b) is done for you in problem 7.12.  Apply those methods when doings parts (a) and (h).  
\newline
\vspace{.15in}
(a)  \[
\begin{aligned} 
(x \vee y) \vee (x \vee -y) \\
&= x \vee y \vee x \vee -y  \;\;  \mbox{Associative Property} \\ % The double backslash ends the line.  The backslash semicolon gives me some space horizontally.  All text within the square bracket will be italicized.  I don't want Demorgan's Law to be italicized.  Therefore, I place it in \mbox{}  
&= x \vee x \vee y \vee -y  \;\; \mbox{Commutative}\\
&= x \vee T \;\; \mbox{Identity}\\
&= \mbox{TRUE} \;\; \mbox{Identity} \\
\square \\
\end{aligned}
\]


\vspace{0.25in}
(b)
\[
\begin{aligned} 
(x \wedge (x \rightarrow y)) \rightarrow y \\
&= [x \wedge (-x \vee y)] \rightarrow y \;\;  \mbox{Translate $\rightarrow$} \\
&= [(x \wedge -x) \vee (x \wedge y)] \rightarrow y \;\; \mbox{Distributive Property}\\ 
&= [\mbox{False} \vee (x \wedge y)] \rightarrow y \;\; \mbox{Identity}\\
&= (x \wedge y) \rightarrow y \;\; \mbox{Identity} \\ 
&= (-(x \wedge y)) \vee y \;\; \mbox{Translate $\rightarrow$} \\
&= (-x \vee -y ) \vee y \;\; \mbox{De Morgans}\\
&= -x \vee (-y \vee y) \;\; \mbox{Associative Property} \\
&= -x \vee \mbox{True} \;\; \mbox{Identity}\\
&= \mbox{TRUE} \;\; \mbox{Identity} \\
\square \\
\end{aligned}
\]
 \vspace{0.25in}

(h) 
\[
\begin{aligned} 
((x \rightarrow y) \wedge (x \rightarrow -y)) \rightarrow -x \\
&= (-x \vee y) \wedge (-x \vee -y)) \rightarrow -x \;\;  \mbox{Translate $\rightarrow$} \\
&= -x \vee (y \wedge -y) \rightarrow -x  \;\; \mbox{Distributive Property}\\ 
&= (-x \vee \mbox{False} ) \rightarrow -x \;\; \mbox{Identity}\\
&= -x \rightarrow -x \;\; \mbox{Identity Elements} \\
&= x \vee -x \;\; \mbox{Translate} \rightarrow\\
&= \mbox{TRUE} \\
\square \\
\end{aligned}
\]

\noindent 8.(6 points) 7.13(a), (c).   Use any method you wish to prove this.  That is, use Theorem 7.2 or truth tables.   
\vspace{.05in} \\
(a) 
\[
\begin{aligned} 
(x \vee y) \wedge (x \vee -y) \wedge -x \\
&= (x \vee (y \wedge -y)) \wedge -x \;\;  \mbox{Distributive Property} \\
&= (x \vee \mbox{False} ) \wedge -x  \;\; \mbox{Identity} \\ 
&= x \wedge -x \;\; \mbox{Identity Elements} \\
&= \mbox{FALSE} \;\; \mbox{Identity}\\
\square \\
\end{aligned}
\]

\vspace{0.25in}
(c) 
\[
\begin{aligned} 
(x \rightarrow y) \wedge ((-x) \rightarrow y) \wedge -y\\
&=  (-x \vee y) \wedge ( x\vee y) \wedge -y\;\;  \mbox{Translate $\rightarrow$} \\
&= (y \vee -x) \wedge (y\vee x) \wedge -y \;\; \mbox{Commutative Property} \\ 
&= y \vee (x \wedge -x) \wedge -y \;\; \mbox{Distributive} \\
&= y \vee \mbox{False} \wedge -y \;\; \mbox{Identity} \\
&= y \wedge -y \;\; \mbox{Identity Elements} \\
&= \mbox{FALSE} \\
\square \\
\end{aligned}
\]




\end{document}


