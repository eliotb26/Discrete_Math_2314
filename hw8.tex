
\documentclass[12pt]{article}
\usepackage[margin=1in]{geometry}
\usepackage{amsmath, amssymb, amsthm, graphicx, hyperref}
\usepackage{enumerate}
\usepackage{fancyhdr}
\usepackage{multirow, multicol}
\usepackage{tikz}
\pagestyle{fancy}
\fancyhead[RO]{ELIOT BROWN Spring 2020}
\fancyhead[LO]{MA-UY 2314: Discrete Mathematics}
\usepackage{comment}
\newif\ifshow
\showfalse
\usepackage{centernot}
\usepackage{cancel}


\begin{document}

\begin{center}
  \textbf{\Large Homework 8} \\
Due: Friday April 10 by 11:59pm 
by Gradescope
\end{center}


\hrule

\vspace{0.25in}

\begin{enumerate}
    \item (3 points) Section 20:  20.12
    Assume $x = N$, where $N= d_k10^{k} +d_{k-1}10^{k-1}+...+d_1 10 + d_0 $\\
    \textbf{Proof by contradiction: }\\
    1. $10|x \rightarrow d_0 \neq 0$\\
    2. $d_0 = 0 \rightarrow 10|x = \textbf{F}$\\
    
    1. \\
    $x=10k$, where $k \in \mathbb{Z}$\\
    $10k = d_k10^{k} +d_{k-1}10^{k-1}+...+d_1 10 + d_0 $\\
    $k = d_k10^{k-1} +d_{k-2}10^{k-2}+...+d_1 + \frac{d_0}{10} $
    since $d_0 \neq 0$, this implies $d_0 \in \{1,2,3,...,9\}$ \\
    but then, $\frac{d_0}{10}$ is not an integer and so k is not an integer and this is a contradiction.
    
    2. \\
    $d_0 = 0$ so $N= d_k10^{k} +d_{k-1}10^{k-1}+...+d_1 10 $ \\
    The statement $10k = d_k10^{k} +d_{k-1}10^{k-1}+...+d_1 10 $ , where $k \in \mathbb{Z}$ is false. \\
    This implies $k = d_k10^{k-1} +d_{k-2}10^{k-2}+...+d_1 $ is false. \\
    However, this is false as $d_1$ through $d_k$ are all integers multiplied by powers of 10, hence the sum of integers is an integer and so this is a contradiction.
    
    
    \item (12 points) Do exam2$\_$beta on Gradescope.  
   
    \item (3 points)  Section 21:  21.5. 
    
    let $X = \{ n \geq 0 : \binom{2n}{n} > 4^n\}$\\
    NTS $X=\emptyset$\\
    \textbf{Proof by WOPT: } Assume $X \neq \emptyset$\\
    Base Case: $n=0$\\
    $\binom{2n}{n} = \binom{0}{0}= \text{ LHS }=0$\\
    $4^n=4^0= \text{RHS}=1$, $0 \leq 1$\\
    This implies P(0) = T \\
    SES:\\
    let k be the smallest element in X s.t.\\
    (i) $k \in X \rightarrow \binom{2k}{k} > 4^k$\\
    (ii) $k \leq x, \;\; \forall x \in X$\\
    $(k-1) \notin X$ according to (ii) and $P(k-1) = T$ according to Base Case.\\
    $\binom{2(k-1)}{k-1} \leq 4^{k-1}$\\
    
    $= \frac{(2k-2)!}{k-1!k-1!} \leq 4^{k-1} $\\
    
    NTS $\frac{(2k)!}{k!k!} \leq 4^{k}$\\
    
    multiply both sides by $\frac{(2k-1)(2k)}{(k)(k)}$\\
    
    $= \frac{(2k-2)!}{k-1!k-1!}\times \frac{(2k-1)(2k)}{(k)(k)}  \leq 4^{k-1} \times \frac{(2k-1)(2k)}{(k)(k)} $\\
    
    $=\frac{(2k)!}{k!k!} \leq 4^{k-1} \times \frac{(2k-1)(2)}{k} $\\
    
    NTS $ 4^{k-1} \times \frac{(2k-1)(2)}{k} \leq 4^k$\\
    
    $= 4^{k-1} \times \frac{(2k-1)(2)}{k} \leq 4^{k-1} \times 4$\\
    
    $= \frac{(2k-1)(2)}{k} \leq 4$\\
    
     $4k-2 \leq 4k \;\;\;  \square$
    
    \item (3 points) Section 21: 21.6\\
    let $X = \{ n \geq 1 :\;\;\;1\times1! + 2\times2! + ... + n\times n! \neq (n+1)!-1\}$\\
    \textbf{NTS} $X=\emptyset$\\
    \textbf{Proof by WOPT: } Assume $X \neq \emptyset$\\
    Base Case: $n=1$\\
    $1\times 1! = \mbox{LHS} = 1$\\
    $(1+1)!-1 = 2! -1 = \mbox{RHS}=1$\\
    So P(1) = T \\
    SES:\\
    let k be the smallest element in X s.t.\\
    (i) $k \in X \rightarrow \;\;\; 1\times1! + 2\times2! + ... + n\times n! \neq (n+1)!-1$\\
    (ii) $k \leq x, \;\; \forall x \in X$\\
    $(k-1) \notin X$ according to (ii) and $P(k-1) = T$ according to Base Case.\\
    
    $1\times1! + 2\times2! + ... + (k-1)\times (k-1)! = ((k-1)+1)!-1$\\
    
    \textbf{NTS} $1\times1! + 2\times2! + ... + k\times k! = (k+1)!-1$\\
    
    add $k\times k!$ to both sides\\
    
    $1\times1! + 2\times2! + ... + (k-1)\times (k-1)! + k\times k! = k!-1 + k\times k!$\\
    
    \textbf{NTS} $k!-1 + k\times k! = (k+1)!-1$\\
    
    $ k\times k! + k! -1 = k!(k+1)-1  $\\
    
    Since $k!(k+1)=(k+1)!$, $k!(k+1)-1 = (k+1)!-1 \;\; \square$ 
    
    
    \item (3 points) Section 21:  21.7.  
    $F_n > 1.6^n , \;\; n\geq29$\\
    let $X = \{ n \geq 29 : F_n \leq 1.6^n$\}\\
    \textbf{NTS} $X=\emptyset$\\
    \textbf{Proof by WOPT: } Assume $X \neq \emptyset$\\
    Base Case: $n=29$\\
    $F_29 = \mbox{LHS}= 832040$\\
    $1.6^{29} = \mbox{RHS} = 830767 $
    so P(29) = T \\
    Base Case: $n=30$\\
    $F_29 > 1.6^{29}$ \\
     
    so P(30) = T \\
    SES:\\
    let k be the smallest element in X s.t.\\
    (i) $k \in X \rightarrow F_n \leq 1.6^n$ \\
    (ii) $k \leq x, \;\; \forall x \in X$\\
    $(k-1) \notin X$ according to (ii) and $P(k-1) = T$ according to Base Case.\\
    
    $F_{k-1} > 1.6^{k-1}$ \;\;\; (iii)\\
    
    then also $k-2$ \\
    $F_{k-2} > 1.6^{k-2}$ \;\;\; (iv) \\ 
    then want to match a side with (i) to substitute so Add (iii) +(iv) 
    $F_{k-1} + F_{k-2} > 1.6^{k-1} + 1.6^{k-2}$ \\ 
    $F_k >  1.6^{k-1} + 1.6^{k-2}$ \;\;\; (v) \\ 
    then Allign with (i) \\ 
    $1.6^{k-1} + 1.6^{k-2} < 1.6^{k}$ \;\;\; then flip to contradict statement \\ 
    NTS: $1.6^{k-1} + 1.6^{k-2} \geq 1.6^{k}$ \\
    divide by $(1.6)^{k-2}$ \\ 
    $1.6 + 1 \geq 1.6^2$ \\ 
    $2.6 \geq 2.56$  \rightarrow \leftarrow \\ 
    $\square$
    
    
%     \textbf{NTS} $F_k > 1.6^k$\\
    
%     multiply $1.6$ on both sides\\
    
%   $F_{k-1} \times 1.6 > 1.6^{k-1} \times 1.6$\\
  
%      $F_{k-1} \times 1.6 > 1.6^{k}$\\
     
%      \textbf{NTS} $F_k \geq F_{k-1} \times 1.6$ \\
     
%      $F_k - F_{k-1} \times 1.6 \geq 0$
     
%      BC $\rightarrow 29 \notin X \rightarrow k \geq 30 \rightarrow F_{30} - F_{29} \times 1.6 > 15005 \rightarrow 15005 \geq 0 \;\; \square$\\
     
    \item (3 points) Section 21:  21.8.\\
    
    $F_0 + F_1 + ... + F_n = F_{n+2} -1 $\\
    let $X = \{ n \geq 0 : F_0 + F_1 + ... + F_n = F_{n+2} -1\}$\\
    \textbf{NTS} $X=\emptyset$\\
    \textbf{Proof by WOPT: } Assume $X \neq \emptyset$\\
    Base Case: $n=0$\\
    $F_0 = \mbox{LHS} = 1$\\
    $F_2 -1 = \mbox{RHS}= 2 - 1=1$\\
    so P(0) = T \\
    SES:\\
    let k be the smallest element in X s.t.\\
    (i) $k \in X \rightarrow F_0 + F_1 + ... + F_n = F_{n+2} -1$ \\
    (ii) $k \leq x, \;\; \forall x \in X$\\
    $(k-1) \notin X$ according to (ii) and $P(k-1) = T$ according to Base Case.\\
    
    $F_0 + F_1 + ... + F_{k-1} = F_{(k-1)+2} -1$\\
    
    $F_0 + F_1 + ... + F_{k-1} = F_{k+1} -1$\\
    
    \textbf{NTS} $F_0 + F_1 + ... + F_k = F_{k+2} -1 $\\
    
    add $F_{k}$ to both sides.\\
    
    $F_0 + F_1 + ... + F_{k-1} + F_{k}  = F_{k+1} + F_{k} -1$\\
    
    
    
    $  F_{k-2} + F_{k-1} = F_{k} \rightarrow F_{k} + F_{k+1} = F_{k+2}$\\
    
    $F_0 + F_1 + ... + F_{k-1} + F_{k}  = F_{k+2} -1 \;\; \square$\\
    
\end{enumerate}





\noindent \textbf{Remark:}  When proving problems from section 21, you must use the Well-Ordering Principle technique.  See page 131 for details.



\end{document}