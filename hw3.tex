\documentclass[12pt]{article}
\usepackage[margin=1in]{geometry}
\usepackage{amsmath, amssymb, amsthm, graphicx, hyperref}
\usepackage{enumerate}
\usepackage{fancyhdr}
\usepackage{multirow, multicol}
\usepackage{tikz}
\pagestyle{fancy}
\fancyhead[RO]{Eliot Brown Spring 2020}
\fancyhead[LO]{MA-UY 2314: Discrete Mathematics}
\usepackage{comment}
\newif\ifshow
\showfalse

\begin{document}

\begin{center}
\ifshow
  \textbf{\Large Homework 0 Solution}\\
\else
  \textbf{\Large Homework 3}\\
\fi
Due: Friday, Feb.. 21, by 11:59pm,\\via Gradescope\\
%Late HW accepted until 9PM on September 8\footnote{For each hour your homework is late, 10 points (out of 100) will be deducted from your homework score.  Please submit your homework on time!}
\end{center}

\hrule

\vspace{0.2cm}
\noindent
\begin{enumerate}[A.]
\item Refer to the Gradescope videos on YouTube if you have questions on how to submit homework.  
\item I refer you to the syllabus for the rubric that is used on the homework.  
\end{enumerate}




\begin{enumerate}
    \item (6 points) Section 8:  8.7(a), 8.9.  Explain your answers. \\ 
    \textbf{Solution 8.7a} $31!$ as order matters with 30 photos in 2 places. As it is not ordered,  \\
    
    \textbf{Solution 8.9} $7!*64$ or 3136 or $7*7*8*8$, as 7 other places in a row, for each row, then 64 places on the board, then using the multiplication rule. \\
    
    
    \item (6 points) Section 8:   8.16, 8.19.  Explain your answers. \\ 
    \textbf{Solution 8.16}  $10*7^3$, first choice * 7 remaining options for the last 3. Consecutive cannot be the same # or next to so its a 4 digit list 0-10. There are 3,430 different combinations. The first number has 10 digits to choose from. Since numbers cannot repeat and numbers cannot be adjacent(a number has 2 adjacent numbers at any given time), this means that the second number has 7 options to choose from. The third number also has 7 options. The fourth number also 7 options. \\
    
    
    \textbf{Solution 8.19} There are 4 cards from the deck and if all are different values then when you pick one the next cannot be the same number, or the suit so you subtract 12, for the suit, and 4 for the same card value. So this means there is $52*36*22*10$ ways. 
    
    
    
    \item (3 points) List 2655 consecutive composite numbers.  Show that they are indeed composite.  \\
    
    \textbf{Solution} NTS: b $\exists$  s.t.  1 < b < 3000! + b  where  $ k = 2, ..., 2657 $ where  b|3000!+b and $x = 3000!+b $ so b|x. $ $ This will then show that all of the values of x are then consecutive composite numbers. \\
    
    
    
    
    \item (3 points) Let $A = 2^{\emptyset}, \; B = 2^{A}, C = 2^{B}$, and $D = 2^{C}$.  Construct D.   
    
    
    
    
    
    \item (18 points) Section 10:  10.1(g), 10.3(c)(d)(e)(f)(g).  
    
    
    
    
    \item (21 points) Section 10: 10.4, 
    
    \textbf{Solution} 
     \noindent A: $2 \epsilon \{1,2,3\}$ \\
     \noindent B: $\subseteq$ \\
     \noindent C: $\epsilon$ \\ 
     \noindent D: $\subseteq$ \\ 
     \noindent E: $\subseteq$ \\
     \noindent F: $\subseteq$ \\
     \noindent G: $\epsilon$ \\
    
    
    
    
    \item (24 points) Section 10: 10.5, 10.6
    \item (9 points) 10.12, 10.14, 10.15.  
    \item (6 points)  Let 
    \[    
    \begin{aligned}
    A &= \{ x \in \mathbb{Z}: x = 6k -5 \mbox{ for some } k \in \mathbb{Z} \} \\
    B &= \{ x \in \mathbb{Z}: x = 3m + 1 \mbox{ for some } m \in \mathbb{Z} \} \\
    \end{aligned}
    \]
    \begin{enumerate}
        \item Is $A \subseteq B$?  If so, prove it.  If not, provide an appropriate counter-example.  Show that your counter-example works, i.e. show that your counterexample belongs to $A$ however does not belong to $B$.  
        \item  Is $B \subseteq A$?  If so, prove it.  If not, provide an appropriate counter-example.  Show that your counter-example works, i.e. show that your counterexample belongs to $B$ however does not belong to $A$.  
    \end{enumerate}
    \end{enumerate}
\end{document}


