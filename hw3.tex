\documentclass[12pt]{article}
\usepackage[margin=1in]{geometry}
\usepackage{amsmath, amssymb, amsthm, graphicx, hyperref}
\usepackage{enumerate}
\usepackage{fancyhdr}
\usepackage{multirow, multicol}
\usepackage{tikz}
\pagestyle{fancy}
\fancyhead[RO]{Spring 2020}
\fancyhead[LO]{MA-UY 2314: Discrete Mathematics}
\usepackage{comment}
\newif\ifshow
\showfalse

\begin{document}

\begin{center}
\ifshow
  \textbf{\Large Homework 0 Solution}\\
\else
  \textbf{\Large Homework 3}\\
\fi
Due: Friday, Feb.. 21, by 11:59pm,\\via Gradescope\\
%Late HW accepted until 9PM on September 8\footnote{For each hour your homework is late, 10 points (out of 100) will be deducted from your homework score.  Please submit your homework on time!}
\end{center}

\hrule

\vspace{0.2cm}
\noindent
\begin{enumerate}[A.]
\item Refer to the Gradescope videos on YouTube if you have questions on how to submit homework.  
\item I refer you to the syllabus for the rubric that is used on the homework.  
\end{enumerate}




\begin{enumerate}
    \item (6 points) Section 8:  8.7(a), 8.9.  Explain your answers.  
    \item (6 points) Section 8:   8.16, 8.19.  Explain your answers.  
    \item (3 points) List 2655 consecutive composite numbers.  Show that they are indeed composite.  
    \item (3 points) Let $A = 2^{\emptyset}, \; B = 2^{A}, C = 2^{B}$, and $D = 2^{C}$.  Construct D.      
    \item (18 points) Section 10:  10.1(g), 10.3(c)(d)(e)(f)(g).  
    \item (21 points) Section 10: 10.4, 
    \item (24 points) Section 10: 10.5, 10.6
    \item (9 points) 10.12, 10.14, 10.15.  
    \item (6 points)  Let 
    \[    
    \begin{aligned}
    A &= \{ x \in \mathbb{Z}: x = 6k -5 \mbox{ for some } k \in \mathbb{Z} \} \\
    B &= \{ x \in \mathbb{Z}: x = 3m + 1 \mbox{ for some } m \in \mathbb{Z} \} \\
    \end{aligned}
    \]
    \begin{enumerate}
        \item Is $A \subseteq B$?  If so, prove it.  If not, provide an appropriate counter-example.  Show that your counter-example works, i.e. show that your counterexample belongs to $A$ however does not belong to $B$.  
        \item  Is $B \subseteq A$?  If so, prove it.  If not, provide an appropriate counter-example.  Show that your counter-example works, i.e. show that your counterexample belongs to $B$ however does not belong to $A$.  
    \end{enumerate}
    \end{enumerate}
\end{document}


