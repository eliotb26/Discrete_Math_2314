
\documentclass[12pt]{article}
\usepackage[margin=1in]{geometry}
\usepackage{amsmath, amssymb, amsthm, graphicx, hyperref}
\usepackage{enumerate}
\usepackage{fancyhdr}
\usepackage{multirow, multicol}
\usepackage{tikz}
\pagestyle{fancy}
\fancyhead[RO]{Spring 2020}
\fancyhead[LO]{MA-UY 2314: Discrete Mathematics}
\usepackage{comment}
\newif\ifshow
\showfalse
\usepackage{centernot}


\begin{document}

\begin{center}
  \textbf{\Large Homework 4} \\
Due: Friday Feb. 28 by 11:59pm
via Gradescope
\end{center}

\hrule

\vspace{0.25in}

\vspace{.25in}

\noindent 1.(57 points)  Section 11.  Problems 11.2, 11.4.  
\vspace{.25in} \\
\textbf{Solution 11.2:} \\
 \item A: There is an integer that is not prime \\
 $\exists x \epsilon \mathbb{Z}$ , x is not prime \\ 
 \item B: For every integer, x is prime and composite \\ 
 $\forall x \epsilon \mathbb{Z}$, x is prime and composite \\
 \item C: For every integer x, the square is not equal to 2. \\
 $\forall x \epsilon \mathbb{Z}$, $x^2 \neq 2$ \\ 
 \item D: There exists a x that is not divisible by 5. \\ 
 $\exists x \epsilon \mathbb{Z} $, not $ 5\nmid x$ \\
 \item E: Every integer x is not divisible by 7. \\
 $\forall x \epsilon \mathbb{Z}, not 7\nmid x$ \\
 
 \item F: There is some integer x, whose square is negative.  \\ 
 $\exists x \epsilon \mathbb{Z}, x^2 < 0$ \\
 
 \item G: There is an integer x, for every integer y, such that xy does not equal 1. \\
 $\exists x \epsilon \mathbb{Z}, \forall y, xy \neq 1$
 
 \item H: For every integer x, and integer y, x divided by y does not equal 10.  \\ 
 $\forall x \epsilon \mathbb{Z} \wedge \forall y \epsilon \mathbb{Z}, x\/ y \neq 10$
 
 \item I: For every integer x there is an integer k when x multiplied by k does not equal 0. \\ 
 $\forall x \epsilon \mathbb{Z}, \exists k \epsilon \mathbb{Z}$, $k*x \neq 0$

\item J: There is some integer x that is greater than or equal to all integers. \\ 
$\exists x \epsilon \mathbb{Z}$, $\forall y \epsilon \mathbb{Z}$, $y \leq x$
\item K: Somebody does not love everybody all of the time. \\ 
$\exists x$, $\forall y$, $\forall t$, x hates y at time t. \\ 


\\
\textbf{Solution 11.4:}
\item A: FALSE
\item B: TRUE
\item C: FALSE
\item D: TRUE
\item E: FALSE
\item F: TRUE
\item G: TRUE
\item H: TRUE \\ 


\noindent 2.(39 points)  Section 11.  Problems 11.5-11.7.
\vspace{.25in} 

\begin{proof} 
\textbf{Problem 11.5}\\
\noindent{A: } \exists x \in \mathbb{Z}, \neg(x < 0) \\
$There exists sometimes integer that is not less than 0$ \\
\noindent{B: } \forall x \in \mathbb{Z}, \neg(x=x+1)\\
$For any integer x, x doesn't equal x+1 $\\
\noindent{C: } \forall x \in \mathbb{N}, \neg(x>10)\\
$For any natural number x , x is not greater than 10$\\
\noindent{D: } \exists x \in \mathbb{N}, \neg(x+x=2x) \\
$There is an integer when added to itself that does not equal a sum two times the number$
\noindent{E: } \forall x \in \mathbb{Z}, \exists y \in \mathbb{Z}, \neg(x>y)\\
$For any integer x, there is an integer y in which x is not greater than y$\\
\noindent{F: } \exists x \in \mathbb{Z}, \exists y \in \mathbb{Z}, \neg(x=y) \\
$There is an integer x and y in which x does not equal y$\\
\noindent{G: } \exists x \in \mathbb{Z}, \forall y \in \mathbb{Z}, \neg(x+y=0)\\
$There is an integer x for any integer y where their sum is not zero.$\\
\end{proof}

\begin{proof}
\textbf{Problem 11.6}\\
The two statements are equivalent because the values came from the same set. The values would be the same if they were interchanged if you changed the statements into words. The next statements are also equivalent for the same reason because the values could be interchangeable if the expressions were written in words.

\end{proof}

\begin{proof}
\textbf{Problem 11.7}\\
\noindent{A: } This is true because the the only positive root of 4 is 2.\\
\noindent{B: } This is false because the square root of 4 is both positive and negative 2\\
\noindent{C: } This is false the square root of of 3 is 1.7 which is not a natural number\\
\noindent{D: } This is true because x=0  \\
\noindent{E: } This is true because the only way xy =y is if x is one. 

\end{proof}

\noindent 3.(21 points)  Section 12 problems 12.1
\vspace{.15in}

\textbf{Solution 12.1}//
\item A. \{1,2,3,4,5,6,7,\}
\item B. \{4,5\}
\item C. \{1,2,3\}
\item D. \{5,6,7\}
\item E. \{1,2,3,6,7\}
\item F. \{(1,4), (1,5), (1,6), (1,7), (2,4), (2,5), (2,6), (2,7), (3,4), (3,5), (3,6), (3,7), (4,4), (4,5), (4,6), (4,7), (5,4), (5,5), (5,6), (5,7) \}
\item G. \{(4,1),(4,2), (4,3), (4,4), (4,5), (5,1), (5,2), (5,3), (5,4), (5,5), (6,1), (6,2), (6,3), (6,4), (6,5), (7,1), (7,2), (7,3), (7,4), (7,5) \} \\ 




\noindent 4.  Prove that 
\[
(A \cup B) - (A \cap B) \subseteq A \triangle B
\]
using the "``"element method.  That is, prove the if-then statement 
\[
x \in (A \cup B) - (A \cap B) \longrightarrow x \in A \triangle B
\]
Your proof should be begin as follows 
\vspace{.15in}

\noindent \textbf{Proof:}  Assume that $x \in (A \cup B) - (A \cap B)$.  
\vspace{.15in}

\Rightarrow  $(x \epsilon A \vee x \epsilon) \wedge (x \notin A \cup x \notin B)$ \\ 
\Rightarrow $ ((x \epsilon A \vee x \epsilon B) \wedge (x \notin A)) \cup (x\epsilon A \vee x \epsilon B) \wedge x \notin B)$ \\ 
\Rightarrow $(( x \epsilon A \wedge x \notin A) \vee ( x\epsilon B \wedge x \notin A)) \vee ((x \epsilon B \wedge x \notin  B) \vee (x \epsilon A \wedge x \notin B))$ \\ 
\Rightarrow $ (F \vee ( x \epsilon B \wedge x \notin A)) \vee (( F \vee (x \epsilon A \wedge x \notin B))$ \\ 
\Rightarrow $x\epsilon (B-A) \vee x \epsilon (A-B)$ // 
\Rightarrow $x \epsilon A \triangle B$ \;\; \square \\



\noindent 5.(6 points)  Section 12 problems 12.14, 12.15.  Prove these statements using the $``$element method".  For instance, let's take a look at 12.14.  To prove $A - \emptyset = A$, we NTS
\begin{enumerate}[(a).]
    \item \textit{$x \in A - \emptyset \longrightarrow x \in A$}
    \item \textit{$x \in A \longrightarrow x \in A - \emptyset$}
\end{enumerate}
In turn, you should have 
\vspace{.15in}

\noindent \textbf{Proof of (a):}  Assume that $x \in A - \emptyset$.  
\vspace{.15in}
\item NTS: A) $x \epsilon A - \emptyset \rightarrow x \epsilon A$ \\ 
\indent B) $x \epsilon A \rightarrow x \epsilon A - \emptyset$
\item Pf: A) Assume $x \epsilon A - \emptyset$ \\
\Rightarrow $x\epsilon A \wedge x \notin \emptyset$ \\
\Rightarrow $x \epsilon A \wedge T$ \\ 
\Rightarrow $x \epsilon A$ \;\; $\square$
\\ 
\item PF: B) Assume $x \epsilon A$ \\ 
\Rightarrow $x \epsilon A = x \epsilon A \vee F$ \\ \Rightarrow $x \epsilon A \vee x \epsilon \emptyset $ \\ 
\Rightarrow $x \epsilon A - \emptyset$ \;\; $\square$ \\

\noindent \textbf{Proof of (b):}  Assume that $x \in A$.  
\vspace{.15in} 

\item NTS: A) $x \epsilon \emptyset - A \rightarrow x \epsilon \emptyset$ \\ 
\indent B) $ x\epsilon \emptyset \rightarrow x \epsilon \emptyset \- A$  \\ 
\item PF (A): Assume $x \epsilon \emptyset - A$ \\ 
\rightarrow $x \epsilon \emptyset \wedge x \notin A$ \\
\rightarrow False $ \wedge x \notin A$ \\ 
\rightarrow False \;\; \square \\ 

\item PF (B): Assume $x \epsilon \emptyset$ \\ 
\rightarrow False \;\; \square \\


\textbf{Solution 12.15}
\item NTS(1): A) $x \epsilon A \triangle A \rightarrow x \epsilon \emptyset$  \\ 
\indent $x \epsilon \emptyset \rightarrow A \triangle A$ \\ 
\item PF (A): Assume $x \epsilon A \triangle A$ \\ 
\rightarrow $x \epsilon (A-A) \cup x \epsilon (A-A)$ \\ 
\rightarrow $x \epsilon \emptyset \cup x \epsilon (A-A)$ \\ 
\rightarrow False \;\; \square \\ 

\item PF (B): Assume $x \epsilon \emptyset$ \\ 
\rightarrow $x \epsilon F$ \;\; $\square$ \\ 

\item NTS(2): A) $x \epsilon A \triangle \emptyset \rightarrow A$ \\ 
\indent B) $x \epsilon A \rightarrow x \epsilon A \triangle \emptyset$ \\ 

\item PF (A) Assume $x \epsilon A \triangle \emptyset$ \\ 
\rightarrow $x \epsilon (A - \emptyset ) \cup ( \emptyset - A)$ \\ 
\rightarrow $(x \epsilon A \wedge x \notin \emptyset) \vee (x \epsilon \emptyset \wedge x \notin A) $ \\ 
\rightarrow $(x \epsilon A \wedge T) \vee (F \wedge x \notin A)$ \\ 
\rightarrow $(X\epsilon A) \vee F$ \\ 
\rightarrow $x \epsilon A$ \;\; $\square$ \\ 

\item PF B): Assume $x \epsilon A$ \\ 
\rightarrow $(x \epsilon A) \vee F$ \\ 
\rightarrow $(x \epsilon A \wedge T) \vee (F \wedge x \notin A)$ \\
\rightarrow $(x \epsilon A \wedge x \notin \emptyset) \vee (x \epsilon \emptyset \wedge x \notin A)$ \\ 
\rightarrow $x\epsilon A \triangle \emptyset$ \;\; \square \\ 




\noindent 6.(24 points)  Section 12 problem 12.21
\vspace{.15in}

\noindent \textbf{Remark:} When doing these problems, I would begin by drawing a Venn diagram associated to the LHS and the RHS (like we did in class). 
\begin{enumerate}
    \item If the Venn Diagram suggest that the statement is false, then you will need to provide a counterexample.  Note that the Venn diagram is not a counterexample.  \textbf{You must justify your counterexample}.  That is, you need to convince me that the LHS $\neq$ RHS for your particular counterexample. 
    \item If the Venn diagram suggest that the statement is true, then you need to prove the statement using the $``$element method".  A Venn diagram is not a proof.  
\end{enumerate}
\noindent A - (B - C) = (A - B) - C is False. CE: Let A = 1,2 B = 2,3 and C = 2,4, then A-(B-C) = 1,2 but (A - B) - C = 1; \\ 
\\
b) (A B) - C = (A - C) - B is True \\
\\
PF 1) Assume $x \in (A-B) - C$ \\
$(x \in A) \wedge (x \notin B) \wedge (x \notin C)$\\
$(x \in A) \wedge (x \notin C) \wedge (x \notin B)$\\
$x \in (A - C) - B$ \\
\\
PF 2) Assume $x \in (A - C) -B$\\
($x \in A) \wedge (x \notin C) \wedge (x \notin B)$ \\
$(x \in A) \wedge (x \notin B)\wedge (x \notin C)$ \\
$x \in (A - B) - C$\\
\\
c) $(A \cup B) - C = (A - C) \cap (B - C)$ is false \\ 
Let A = {1,2}, B = {2,3} C = {2,4}\\
$(A \cup B) - C =  \{1,2,3\} - \{2,4\}  = \{1,3\}$\\
(A - C)$ \cap (B - C) = $ \{1\} \cap {3}$ = \emptyset$\\
{1,3} $ \neq \emptyset$ \\ 
\\
d) if $A = B - C$, then $B = A \cup C$ is false \\
Let A = \{2,3,4\}, B = \{2,3,4\}, C = \{1\} \\
$A = B - C = \{2,3,4\}$ \\
$A \cup C = \{1,2,3,4\}$ \\
$\{1,2,3,4\} \neq \{2,3,4\}$ \\

\\

e) if B= A \cup C, then A= B-C is false\\
let A= \{1,2,3\} , C= \{3,4,5\} \\
A \cup C = B = \{1,2,3,4,5\} \\
B - C = \{1,2\} \\
$B-C \neq A \;\; \square$ \\

f) $|A-B| = |A|-|B|$ is false. \\
let A=1 and B=-5 \\
$|A-B| = 6 $\\
$|A|-|B|=1-5=-4$ \\
$|A-B| \neq |A|-|B| \;\; \square$\\

g) (A-B) \cup B = A is False.\\
let A= \{1,2,3\} and let B= \{4,5\} \\
A - B = \{1,2,3\} \\
(A-B) \cup B = \{1,2,3,4,5\} \\
$(A-B) \cup B \neq A \;\; \square$



h) (A \cup B)-B = A is False.\\
Let A= \{1,2,3,4,5\} and let B= \{4,5\} \\
A \cup B = \{1,2,3,4,5\} \\
(A \cup B) - B = \{1,2,3\} \\
$(A \cup B)-B \neq A \;\; \square$\\
\vspace{0.15in}


\noindent 7.(6 points) 
\begin{enumerate}[(a).]
    \item Prove that 
\[
A - (B \cup C) \subseteq (A -B) \cap (A - C)
\]
using the $``$element method".  That is, NTS
\[
x \in A - (B \cup C) \longrightarrow x \in (A - B) \cap (A - C)
\]
Your proof you start as follows
\vspace{.15in}
\noindent \textbf{Proof:} Assume that $x \in A - (B \cup C).$
\begin{proof}
\textbf{Proof} Assume that $x \in A - (B \cup C).$\\
\Rightarrow $ x \in (A-B) \cap (A-C)$ \\
\Rightarrow $ x \in A \wedge (x \notin (B \cup C))$\\
\Rightarrow $ x \in A \wedge -(x \in (B \cup C))$\\
\Rightarrow $ x \in A \wedge (x \notin B \wedge x \notin C)$
\Rightarrow $ x \in A \wedge x \in A \wedge x \notin B \wedge x \notin C$ \\
\Rightarrow $ x \in A \wedge x \notin B \wedge x \in A \wedge x \notin C$ \\
\Rightarrow $ x \in (A-B) \wedge x \in (A-C)$ \\
\Rightarrow $ (A-B) \cap (A-C)$ \\

\end{proof}






\noindent \textbf{Proof:} Assume that $x \in A - (B \cup C).$
\item Prove that 
\[
(A - B) \cup (A - C) \subseteq A - (B \cap C)
\]
again using the $``$element method".  
\end{enumerate}

\noindent \textbf{Remark:} Note that problem 7 is taken from Proposition 12.12 on page 63 (DeMorgan's law).  In proving 7(a), you have proven half of the statement 
\[
A - (B \cup C) =  (A -B) \cap (A - C)
\]
In proving 7(b), you have proven half of the statement 
\[
(A - B) \cup (A - C) = A - (B \cap C)
\]

\begin {proof}
\textbf{Proof } Assume $x \in (A-B) \cup (A-C)$\\
\Rightarrow x \in A \wedge x\notin (B \cap C) \\
\Rightarrow x \in A \wedge (x\notin B) \cup (x\notin C) \\
\Rightarrow x \in A \wedge x \in A \wedge ((x\notin B) \cup (x\notin C)) \\
\Rightarrow x \in A \wedge (x\notin B) \cup (x\notin C) \\
\Rightarrow (A-B) \cup (A-C)\\

\end{proof}


\noindent 8.(3 points)  Section 12 Problem 12.24.  Prove this problem using Proposition 12.4 on page 58.  
\vspace{.15in}

\item Assume $x \epsilon |A \cup B \cup C|$ \\ 
\[
\begin{aligned} 
 |A \cup B \cup C|\\
&= |A \cup Z|  \;\;\;\;  \mbox{Z = $B \cup C$} \\   
&= |A| + |Z| - |A \cap Z|  \;\; \mbox{Definition of Union}\\
&= |A| + |B \cup C| - |A \cap (B \cup C)| \;\; \mbox{}\\
&= |A| + |B| + |C| - |B \cap C| - |(A \cap C) \cup (A \cap B)| \;\; \mbox{Distributive} \\
&= |A| + |B| + |C| - |B \cap C| - |A \cap C| - |A \cap C| + |A \cap A \cap B \cap C| \;\;\; \mbox{Definition of Union} \\
&= |A| + |B| + |C| - |B \cap C| - |A \cap C| - |A \cap C| + |A \cap A \cap B \cap C| \\ 
\square \\
\end{aligned}
\]



\\
\noindent 9.(24 points)  Section 14 problems 14.1, 14.2. \\
\noindent 
 14.1  \\
\indent a) (1,2), (1,3), (1,4), (1,5), (2,3), (2,4), (2,5), (3,4), (3,5), (4,5) \\
\indent b) (1,1), (1,2), (1,3), (1,4), (1,5), (2,2) (2,4), (3,3), (4,4), (5,5) \\
\indent c) (1,1), (2,2), (3,3), (4,4), (5,5) \\
\indent d) (1,1), (2,2), (3,3), (4,4), (5,5), (1,2), (1,3), (1,4), (1,5), (2,1), (3,1), (4,1),(5,1), (2,3), (2,4), (2,5), (4,2), (3,2), (4,2), (5,2), (3,4), (3,5), (4,3), (5,3), (4,5), (5,4)\\

14.2 \\
\indetn a) The consecutive relation \\
\indent b) The greater-than-or-equal-to relation \\
\indent c) The equal-to-6 relation \\
\indent d) The is-divisible relation\\

\vspace{.15in}

\noindent 10.(33 points) Section 14 problems 14.3 and 14.4. \\

14.3 \\ 
\indent a) Reflexive: TRUE. contains (a, a) for every ∈ a there exists an R\\
\indent Ir-reflexive: False. x = 1, 1R1 = False\\
\indent Symmetric: TRUE. (a, b) there exists an  R and a = b.So(b, a) ∈ R \\
\indent Anti-Symmetric: True. R only contains like (a, a) \\
\indent Transitive: TRUE. \\
\\
\indent b) Reflexive: False: 1R1 = F \\
\indent Ir-reflexive: True\\
\indent Symmetric: False (1,2) 1R2 implies 2R1 = F \\ 
\indent Anti-Symmetric: True \\
\indent Transitive: False; (1,2) there is no (2,1) \\
\\
\indent c)Reflexive: False (2,2)  is not in R \\ 
\indent Ir-reflexive: False (1,1) exists in R \\
\indent Symmetric: False (1,2) exists in R, (2,3) there exists in R, (1,3) does not exist in R \\ 
\indent Anti-Symmetric: TRUE  \\ 
\indent Transitive: TRUE \\ 
\\ 
\indent d) Reflexive: False (2,2) doesn't exist in R\\
\indent Ir-reflexive:  False, x = 1, 1r1 = False \\
\indent Symmetric: TRUE\\
\indent Anti-Symmetric: False (1,2) exists and so does (2,1) \\
\indent Transitive: False (3,4) exists, (4,3) exists but (3,3) does not exist. \\
\\
\indent e) Reflexive: TRUE
\indent Ir-reflexive: False (1,1) exists in R\\
\indent Symmetric: TRUE \\
\indent Anti-Symmetric: False (1,2) exists in R and (2,1) exists in R\\
\indent Transitive: TRUE\\
\\
14.4\\
\indent a) Reflexive, symmetric, Transitive \\
\indent b) Ir reflexive, Anti-symmetric \\
\indent c) Reflexive, symmetric,Transitive \\
\indent d) Reflexive, symmetric \\
\indent e) Irreflexive, symmetric \\
\indent f) Irreflexive, Anti-symmetric \\

\vspace{.15in}

\noindent \textbf{Remark:} When doing problem 14.3:  If a particular property fails, justify why with an appropriate counterexample.  For instance, when looking at 14.3(a), we notice that \textit{Irreflexive}=F.  A counterexample would be say $x = 1$ since $1 \centernot{R} 1$ = F.  








\end{document}